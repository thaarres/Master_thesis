\newpage
\thispagestyle{empty}
\vspace*{2.0cm}


\begin{center}
{\bf Abstract}
\end{center}

In the new energy regime explored by the Large Hadron Collider new heavy resonances suggested by extensions of the Standard model become available for production and can be detected with the CMS detector. Of recent interest are resonances that decay into Higgs bosons, leading to final states with highly energetic (``boosted'') Higgs bosons that decay dominantly to b quarks. Due to the high momentum of the Higgs boson, the b quarks from its decay are highly collimated, such that they cannot be resolved by the jet algorithm and are merged into a single jet. Current methods to
identify jets originating from a b quark decay ("b-tagging") fail in the case when the two b quarks are too collimated, because they are not treating the overlap of the tracks initiated by the two b quarks correctly. A dedicated
b-tagging algorithm for jets containing two b quarks is therefore needed. This thesis presents the first b-tagging algorithm in CMS capable of tagging jets where two b quarks are merged into one jet. It is capable of discriminating between jets containing one b quark and jets containing two, as well as discriminating two-b jets from light-flavoured jets. The algorithm also provides discrimination between two-b jets coming from a boosted Higgs boson and two-b jets stemming from the QCD background. The presented algorithm achieves a 30 and 10\% higher identification efficiency for highly boosted Higgs bosons than the current methods at same mistagging probability for two-b versus one b jet and two-b versus light-flavoured jet discrimination respectively.

\cleardoublepage






























































